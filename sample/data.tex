% version du document
% \version{Mémoire provisoire}

% intitulé du doctorat
\doctorate{Doctorat ParisTech}

% école doctorale
\doctorateschool{École doctorale numéro 001 : École la plus classe du monde}

% école
\institute{l'École Nationale Supérieure d'Arts et Métiers}

% logo de l'école (si différent de ENSAM)
% \institutelogo{logo}

% couleur de l'école (si différent de ENSAM)
% \institutecolor{model}{color-spec}

% spécialité
\speciality{Classe}

% nom du doctorant
\author{George Abitbol}

% titre de la thèse affiché en couverture
\title{Comment faire un grand détournement}

% numéro de la thèse
\thesisid{2018-ENAM-0000}

% directeur(s) de thèse
% s'il y en a plusieurs ajouter [p]
\director[]{Michel Hazanavicius}

% encadrant(s) de thèse
% s'il y en a plusieurs ajouter [p]
\supervizor[]{Dominique Mézerette}

% date de soutenance
\date{1\textsuperscript{er} janvier 2018}

% composition du jury
% trier par ordre alphabétique pour le mémoire provisoire
% puis par rôle pour le mémoire final
\jury{%
  \jurymember{Orson Welles}{Professeur, la Sorbonne}{Président}
  \jurymember{James Stewart}{Professeur, Université de Neuf Heures}{Rapporteur}
  \jurymember{Clark Gable}{Professeur, la Sorbonne}{Examinateur}
}

% adresse
\address{%
  Arts et Métiers ParisTech -- Campus du Texas
}{%
  EA 00 -- Cananac -- Institut de la Classe
}

% titre en français
\titlefr{Comment faire un grand détournement}

% résumé en français
\abstractfr{\lipsum[1]}

% mots clés en français
\keywordsfr{flim, cyclimse, détournement, classe}

% titre en anglais
\titleen{How to do a ``grand détournement''}

\abstracten{\lipsum[2]}

\keywordsen{class, man, top, pop}
